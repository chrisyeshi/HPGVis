% $Id: template.tex 11 2007-04-03 22:25:53Z jpeltier $

\documentclass{vgtc}                          % final (conference style)
% \documentclass[review]{vgtc}                 % review
%\documentclass[widereview]{vgtc}             % wide-spaced review
%\documentclass[preprint]{vgtc}               % preprint
% \documentclass[electronic]{vgtc}             % electronic version

%% Uncomment one of the lines above depending on where your paper is
%% in the conference process. ``review'' and ``widereview'' are for review
%% submission, ``preprint'' is for pre-publication, and the final version
%% doesn't use a specific qualifier. Further, ``electronic'' includes
%% hyperreferences for more convenient online viewing.

%% Please use one of the ``review'' options in combination with the
%% assigned online id (see below) ONLY if your paper uses a double blind
%% review process. Some conferences, like IEEE Vis and InfoVis, have NOT
%% in the past.

%% Figures should be in CMYK or Grey scale format, otherwise, colour 
%% shifting may occur during the printing process.

%% These three lines bring in essential packages: ``mathptmx'' for Type 1 
%% typefaces, ``graphicx'' for inclusion of EPS figures. and ``times''
%% for proper handling of the times font family.

\usepackage{mathptmx}
\usepackage{graphicx}
\usepackage{times}

%% We encourage the use of mathptmx for consistent usage of times font
%% throughout the proceedings. However, if you encounter conflicts
%% with other math-related packages, you may want to disable it.

%% If you are submitting a paper to a conference for review with a double
%% blind reviewing process, please replace the value ``0'' below with your
%% OnlineID. Otherwise, you may safely leave it at ``0''.
\onlineid{118}

%% declare the category of your paper, only shown in review mode
\vgtccategory{Research}

%% allow for this line if you want the electronic option to work properly
\vgtcinsertpkg

%% In preprint mode you may define your own headline.
%\preprinttext{To appear in an IEEE VGTC sponsored conference.}

%% Paper title.

\title{Feature Tracking on In Situ Generated Explorable Images}

%% This is how authors are specified in the conference style

%% Author and Affiliation (single author).
%%\author{Roy G. Biv\thanks{e-mail: roy.g.biv@aol.com}}
%%\affiliation{\scriptsize Allied Widgets Research}

%% Author and Affiliation (multiple authors with single affiliations).
\author{Yucong (Chris) Ye\thanks{e-mail: chrisyeshi@gmail.com} %
\and Yang Wang\thanks{e-mail: gnavvy@gmail.com} %
\and Robert Miller\thanks{e-mail: bobmiller@ucdavis.edu} %
\and Kwan-Liu Ma\thanks{e-mail: ma@cs.ucdavis.edu}}
\affiliation{\scriptsize VIDi Lab \\ University of California, Davis}

%% Author and Affiliation (multiple authors with multiple affiliations)
% \author{Roy G. Biv\thanks{e-mail: roy.g.biv@aol.com}\\ %
%         \scriptsize Starbucks Research %
% \and Ed Grimley\thanks{e-mail:ed.grimley@aol.com}\\ %
%      \scriptsize Grimley Widgets, Inc. %
% \and Martha Stewart\thanks{e-mail:martha.stewart@marthastewart.com}\\ %
%      \parbox{1.4in}{\scriptsize \centering Martha Stewart Enterprises \\ Microsoft Research}}

%% A teaser figure can be included as follows, but is not recommended since
%% the space is now taken up by a full width abstract.
%\teaser{
%  \includegraphics[width=1.5in]{sample.eps}
%  \caption{Lookit! Lookit!}
%}

%% Abstract section.
\abstract{
Scientific simulations have been a powerful tool used by scientists to solve different complex problems. It has been a common practice to output all the raw simulation data to storage and then in depth analysis of the raw outputs are performed at a later time. However, as we are entering peta- and exa-scale computing, it is no longer feasible to store all the raw simulation data. Instead, in situ visualization/analysis is a promising way to encounter the I/O bottleneck and reduce the storage requirement. In this paper, we provide a complete system to generate and interact with explorable images in situ. With explorable images, scientists are able to explore the transfer function space of the original volume. Furthermore, our system is able to conduct feature extraction and tracking both forward and backward through time. Finally, our system allows the scientists to steer the visualization with new parameters while the simulation is running.
} % end of abstract

%% ACM Computing Classification System (CCS). 
%% See <http://www.acm.org/class/1998/> for details.
%% The ``\CCScat'' command takes four arguments.

\CCScatlist{
  \CCScat{I.3.2}{Computer Graphics}{Graphics Systems}{Distributed/network graphics}
  \CCScat{I.3.7}{Computer Graphics}{Three-Dimensional Graphics and Realism}{Color, shading, shadowing, and texture};
}

%% Copyright space is enabled by default as required by guidelines.
%% It is disabled by the 'review' option or via the following command:
% \nocopyrightspace

%%%%%%%%%%%%%%%%%%%%%%%%%%%%%%%%%%%%%%%%%%%%%%%%%%%%%%%%%%%%%%%%
%%%%%%%%%%%%%%%%%%%%%% START OF THE PAPER %%%%%%%%%%%%%%%%%%%%%%
%%%%%%%%%%%%%%%%%%%%%%%%%%%%%%%%%%%%%%%%%%%%%%%%%%%%%%%%%%%%%%%%%

\begin{document}

%% The ``\maketitle'' command must be the first command after the
%% ``\begin{document}'' command. It prepares and prints the title block.

%% the only exception to this rule is the \firstsection command
\firstsection{Introduction}

\maketitle

%% \section{Introduction}

Nowadays, scientific simulations are getting larger and larger. In the near future, there will be no way to output raw data because it will be too big for any possible storage. As a result, we need to do visualization/analysis in situ so that we can more precisely get what we want from a simulation. As a consequence, reduce the data size.

Explorable image is really good for the data intensive computing era. explorable images allow the simulations to only store really small data yet allow users to explore the original volume in the transfer function domain.

Furthurmore, we extend on the original explorable image and allow the users to perform feature tracking based on explorabe images. Feature tracking in explorable images allows us to both track forward and backward over time. If we track in the simulation, we can only track forward.

\section{Related Work}

\cite{proxy,anna_eg_2010,anna_pv_2010,ma2007situ,hf_2010}

\section{Explorable Image}

What we did with the original explorable image?

First of all, this one is a distributed rendering, thanks to Hongfeng's code. Using 2-3 swap.

This is a software rendering. Can be used in situ with simulations. No extra renderer was used (like Mesa3D).

Changed to rendering isosurfaces instead of continuous opacity map. Which means in each bin of the RAF, only the average value is used toward attenuation. The reason is people like to use thin slabs in a transfer function to create semi-transparent surfaces.

\section{Feature Tracking}

We are using the depth map to perform feature tracking. There are 16 bins and each bin has its own depth value. Based on the depth map, we can distinguish features if the depth values are discontinuous.

Furthurmore, we can use the neighbor depth values of the current pixel to calculate the gradiant and use that as a more accurate detection of edges.

\section{Results}

\section{Discussion}

\section{Future Work}

\begin{equation}
 \sum_{j=1}^{z} j = \frac{z(z+1)}{2}
\end{equation}

\begin{table}
  \caption{Vis Paper Acceptance Rate}
  \label{vis_accept}
  \scriptsize
  \begin{center}
    \begin{tabular}{cccc}
      Year & Submitted & Accepted & Accepted (\%)\\
    \hline
      1994 &  91 & 41 & 45.1\\
      1995 & 102 & 41 & 40.2\\
      1996 & 101 & 43 & 42.6\\
      1997 & 117 & 44 & 37.6\\
      1998 & 118 & 50 & 42.4\\
      1999 & 129 & 47 & 36.4\\
      2000 & 151 & 52 & 34.4\\
      2001 & 152 & 51 & 33.6\\
      2002 & 172 & 58 & 33.7\\
      2003 & 192 & 63 & 32.8\\
      2004 & 167 & 46 & 27.6\\
      2005 & 268 & 88 & 32.8\\
      2006 & 228 & 63 & 27.6
    \end{tabular}
  \end{center}
\end{table}

\begin{figure}[htb]
  \centering
  \includegraphics[width=1.5in]{sample.eps}
  \caption{Sample illustration.}
\end{figure}

%% if specified like this the section will be ommitted in review mode
\acknowledgements{
The authors wish to thank A, B, C. This work was supported in part by
a grant from XYZ.}

\bibliographystyle{abbrv}
%%use following if all content of bibtex file should be shown
%\nocite{*}
\bibliography{template}
\end{document}
